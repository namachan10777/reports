\documentclass[dvipdfmx]{jsarticle}
\usepackage{here}
\usepackage[dvipdfmx]{graphicx}
\begin{document}
	\title{順序回路}
	\author{3EC 28 中野将生}
	\maketitle
	\section{目的}
		順序回路(sequential circuit)は、組み合わせ回路の持つ次の問題に対処するものである。
		\begin{enumerate}
			\item 実現したい処理内容によっては非常に大規模な回路になる。
			\item 回路の出力が現在の入力だけで決まり、入力の履歴や記憶に関する処理が出来ない。
		\end{enumerate}
		1.の問題を解決する為には、処理を段階ごとに分割して順次解いていく型式に変更すればよい。そのためには、演算を段階毎に順次実行できる回路が必要になるが、
		このような回路を順序回路と呼ぶ。 \par
		順序回路の実現や2.の問題の解決にあ、出力が現在の入力だけでなく過去の入力(=現在の回路の状態)に依存する論理素子、つまり「記憶」を扱える論理素子が必要になる。
		このような回路要素がフリップフロップ回路(Flip・Flop Circuit)であり、その応用としてシフトレジスタ(Shift Register)がある。\par
		ここでは、前回までに学習した論理素子を用いて1ビット記憶装置が作成できること、またそのような動作原理について理解すると共に、
		その応用であるシフトレジスタについての理解を深めることを目的とする。
	\section
\end{document}
