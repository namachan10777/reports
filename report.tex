\documentclass[dvipdfmx]{jsarticle}
\usepackage{here}
\begin{document}
	\title{組み合わせ回路と順序回路}
	\author{3EC 中野 将生}
	\maketitle
	\section{目的}
		本実験では論理回路実習装置(IFF-02)を用い、順序回路と組み合わせ回路の実習を行う。
		第一周では、基本的な論理演算解論についての実習を行い、論理素子の動作の確認を行うと共に、
		その応用として加算回路の実習を行い、組み合わせ回路についての理解を深めることを目的とする。
		\section{実習方法}
			\subsection{操作方法}
				\subsubsection{操作方法}
					\begin{itemize}
						\item 電源部(POWER) \\
							本装置の電源供給に関する操作箇所
						\item パルス発生器(PULSE GENERATOR)
							\begin{itemize}
								\item リセットパルス(RESET PULSE) \\
									プッシュスイッチを押している時間だけのパルス幅で、$A$端子からは正の、$\bar{A}$端子からは負の波形が出力される
								\item 単発クロックパルス(MANUAL CLOCK PULSE) \\
									プッシュスイッチを押すことにより、パルス幅約 $0.3 \mu sec$のパルスが、
									$A$端子からは正の、$\bar{A}$端子からは負の波形が出力される。
								\item クロックパルス (CLOCK PULSE)
									1 kHz端子からは1 kHzの、10 kHz端子からは10 kHzの方形波が出力される。
							\end{itemize}
						\item GND
							測定用アース端子。出力信号をオシロスコープ等で測定するときに、測定用アース端子を接続する。
						\item 表示器
							7セグメントタイプのLED表示器
							入力レベルが「1」のときは"1"、「0」のときは"0"と表示される。
					\end{itemize}
				\subsubsection{回路素子及び各回路素子の入・出力端子}
					\begin{itemize}
						\item 論理積(AND)
						\item 論理和(OR)
						\item 否定(NOT)
						\item 論理積の否定(NAND)
						\item 論理和の否定(NOR)
						\item エンコーダ(ENCODER) \\
						5入力OR1個、4入力OR4個、2入力OR1個から構成されている10進→2進の変換回路。
						\item デコーダ(DECODER) \\
						4入力AND10個で構成されている2進→10進の変換回路。
						\item 全加算回路 \\
						Exclusive ORを使用した全加算(FULL ADDER)回路。
						半加算回路またはExclusive ORとしても使用できる。
						\item R-Sフリップフロップ(R-S FLIP-FLOP)
						\item J-Kフリップフロップ(R-S FLIP-FLOP)
						\item シフトレジスタ(SHIFT REGISTER)
					\end{itemize}
				\subsubsection{操作手順}
					\begin{enumerate}
						\item 電源スイッチをOFFにする。
						\item 各実習項目により、パネル上の論理回路を使用して回路を構成する。
						\item 電源スイッチをONにする
						\item 各実習項目の実習を行う。入力レベルは設定スイッチにより設定する。
						\item 実習が終了したら電源スイッチをOFFにした後、接続を外す。
					\end{enumerate}
			\subsection{組み合わせ回路の実習}
				組み合わせ回路(Combinational Logic Circuit)は、出力が入力だけに関係する回路で、
				基本となる素子として、論理回路(AND)、論理和(OR)、否定(NOT)、論理積の否定(NOT)、論理積の否定(NAND)、論理和の否定(NOR)、などがあり、
				その応用としてExclusive OR、半加算器、全加算器(FULL ADDER)、エンコーダ、デコーダなどがある。
				\subsubsection{基本論理回路の実習}
					組み合わせ回路の基本実習として、論理積、論理和、否定、論理和の否定を実習する。
					各素子の入力に対する出力を表示器でモニタすることにより、素子の動作確認を行う。
					\begin{enumerate}
						\item 論理積(AND) \\
							論理積は、$Y=A \cdot B$で表現され、入力AとBがいずれも「1」のときのみ、出力Yが"1"、
							他の条件では全て"0"となるもので、この式を満足する論理回路をANDと呼ぶ。
							パネル上のAND素子を使用して回路を構成し、入力A,B及び出力Yを表示器に接続して動作確認を行う。
							\begin{itemize}
								\item 実習1 \\
									AND回路の実習を行い、真理地表を作成しなさい
								\item 真理値表 \\
									\begin{table}[H]
										\center
										\caption{AND回路の真理値表の作成例 \label{tb:and}}
										\begin{tabular}{|c|c|c|}
											\hline
											A & B & Y \\ \hline
											0 & 0 & 0 \\ \hline
											0 & 1 & 0 \\ \hline
											1 & 0 & 0 \\ \hline
											1 & 1 & 1 \\ \hline
										\end{tabular}
									\end{table}
							\end{itemize}
						\item 論理和(OR) \\
							論理積は、$Y=A + B$で表現され、入力AとBがいずれも「0」のときのみ、出力Yが"0"、
							他の条件では全て"1"となるもので、この式を満足する論理回路をORと呼ぶ。
							パネル上のOR素子を使用して回路を構成し、表示器により動作確認を行う。
							\begin{itemize}
								\item 実習2 \\
									OR回路の実習を行い、真理地表を作成しなさい
								\item 真理値表 \\
									\begin{table}[H]
										\center
										\caption{OR回路の真理値表の作成例 \label{tb:or}}
										\begin{tabular}{|c|c|c|}
											\hline
											A & B & Y \\ \hline
											0 & 0 & 0 \\ \hline
											0 & 1 & 1 \\ \hline
											1 & 0 & 1 \\ \hline
											1 & 1 & 1 \\ \hline
										\end{tabular}
									\end{table}
							\end{itemize}
						\item 否定(NOT) \\
							否定は、インバータとも言われ、$Y=\bar{A}$で表現され、入力と出力の関係は常に正反対になり、
							この式を満足する回路をNOT回路と呼ぶ。
							パネル上のNOT素子を使用して回路を構成し、表示器により動作確認を行う。
							\begin{itemize}
								\item 実習3 \\
									NOT回路の実習を行い、真理地表を作成しなさい
								\item 真理値表 \\
									\begin{table}[H]
										\center
										\caption{NOT回路の真理値表の作成例 \label{tb:not}}
										\begin{tabular}{|c|c|}
											\hline
											A & Y \\ \hline
											0 & 1 \\ \hline
											1 & 0 \\ \hline
										\end{tabular}
									\end{table}
							\end{itemize}
						\item 論理積の否定(NAND) \\
							論理積の否定は、$Y=\overline{A \cdot B}$で表現され、入力AとBがいずれも「1」のときのみ、出力Yが"0"、
							他の条件では全て"1"となるもので、この式を満足する論理回路をNAND回路と呼ぶ。
							パネル上のNAND素子を使用して回路を構成し、表示器により動作確認を行う。
							\begin{itemize}
								\item 実習4 \\
									NAND回路の実習を行い、真理地表を作成しなさい
								\item 真理値表 \\
									\begin{table}[H]
										\center
										\caption{NAND回路の真理値表の作成例 \label{tb:nand}}
										\begin{tabular}{|c|c|c|}
											\hline
											A & B & Y \\ \hline
											0 & 0 & 1 \\ \hline
											0 & 1 & 1 \\ \hline
											1 & 0 & 1 \\ \hline
											1 & 1 & 0 \\ \hline
										\end{tabular}
									\end{table}
							\end{itemize}
						\item 論理和の否定(NOR) \\
							論理和の否定は、$Y=\overline{A + B}$で表現され、入力AとBがいずれも「0」のときのみ、出力Yが"1"、
							他の条件では全て"0"となるもので、この式を満足する論理回路をNOR回路と呼ぶ。
							パネル上のNOR素子を使用して回路を構成し、表示器により動作確認を行う。
							\begin{itemize}
								\item 実習5 \\
									NOR回路の実習を行い、真理地表を作成しなさい
								\item 真理値表 \\
									\begin{table}[H]
										\center
										\caption{NOR回路の真理値表の作成例 \label{tb:nor}}
										\begin{tabular}{|c|c|c|}
											\hline
											A & B & Y \\ \hline
											0 & 0 & 1 \\ \hline
											0 & 1 & 0 \\ \hline
											1 & 0 & 0 \\ \hline
											1 & 1 & 0 \\ \hline
										\end{tabular}
									\end{table}
							\end{itemize}
					\end{enumerate}
				\subsubsection{組み合わせ回路の実習}
					基本素子を複数接続して論理回路を構成し、その式と動作を確認する応用動作としてド・モルガンの定理証明、
					排他的論理和、加算器及びエンコーダ・デコーダを実習する。
					\begin{enumerate}
						\item ド・モルガンの定理証明 \\
							ド・モルガンの定理は、式\ref{eq:demorgan1}、及び式\ref{eq:demorgan2}で示される。
							\begin{eqnarray}
								\label{eq:demorgan1}
								\overline{A \cdot B} &=& \bar{A} + \bar{B} \\
								\label{eq:demorgan2}
								\overline{A + B} &=& \bar{A} \cdot \bar{B}
							\end{eqnarray}
							この式\ref{eq:demorgan1}、及び式\ref{eq:demorgan2}を証明するのであるが、
							上式を書き直すと、式\ref{eq:demorgan3}及び式\ref{eq:demorgan4}になる。
							\begin{eqnarray}
								\label{eq:demorgan3}
								Y_1 &=& \overline{A \cdot B} \\
								Y_2 &=& \bar{A} + \bar{B} \\
								Y_1 &=& Y_2 \\
							\end{eqnarray}
							\begin{eqnarray}
								\label{eq:demorgan4}
								Y_3 &=& \overline{A + B} \\
								Y_3 &=& \bar{A} \cdot \bar{B} \\
								Y_3 &=& Y_4
							\end{eqnarray}
							式\ref{eq:demorgan3}、\ref{eq:demorgan4}の論理回路をそれぞれパネル上で構成して動作確認を行い、
							$Y_1=Y_2$,$Y_3=Y_4$であれば証明が成立したと言う方法を取る。
							\begin{itemize}
								\item 実習6 \\
									式\ref{eq:demorgan3}を証明する回路を構成して動作確認を行い、真理地表を作成して、
									ド・モルガンの定理式\ref{eq:demorgan3}を証明しなさい。
								\item 真理値表
									\begin{table}[H]
										\center
										\caption{ド・モルガンの定理証明1 \label{tb:demorgan1}}
										\begin{tabular}{|c|c|c|c|}
											\hline
											A & B & $Y_1$ &$Y_2$ \\ \hline
											0 & 0 & 1 & 1 \\ \hline
											0 & 1 & 1 & 1\\ \hline
											1 & 0 & 1 & 1\\ \hline
											1 & 1 & 0 & 0\\ \hline
										\end{tabular}
									\end{table}
								\item 実習7 \\
									式\ref{eq:demorgan4}を証明する回路を構成して動作確認を行い、真理地表を作成して、
									ド・モルガンの定理式\ref{eq:demorgan4}を証明しなさい。
								\item 真理値表
									\begin{table}[H]
										\center
										\caption{ド・モルガンの定理証明2 \label{tb:demorgan2}}
										\begin{tabular}{|c|c|c|c|}
											\hline
											A & B & $Y_1$ &$Y_2$ \\ \hline
											0 & 0 & 1 & 1 \\ \hline
											0 & 1 & 0 & 0\\ \hline
											1 & 0 & 0 & 0\\ \hline
											1 & 1 & 0 & 0\\ \hline
										\end{tabular}
									\end{table}
							\end{itemize}
						\item 排他的論理和(Exclusive OR) \\
							排他的論理和は、$Y = \bar{A} \cdot B + A \cdot \bar{B} = A \oplus B$で表現され、
							入力AとBが同じレベルのとき、出力Yが"0"、異なるレベルの時は"1"となるもので、この式を満足する論理回路をExclusive OR(EXOR)と言う。
							\begin{itemize}
								\item 実習8\\
									基本論理素子を用いてEXOR回路を設計しなさい。また動作確認を行い、真理値表を作成しなさい。
							\end{itemize}
						\item 加算器(ADDER) \\
							一般に加算器とは2つの入力の和とその最上位からの桁上がりを求める回路である。
							ここでからの桁上がりを考慮する回路を全加算回路、(Full Adder)と言う。
							\begin{itemize}
								\item 半加算器、全加算器の回路をそれぞれ構成しなさい。また動作確認を行い、真理地表を作成しなさい。
							\end{itemize}
						\item エンコーダ(ENCODER)、デコーダ(DECODER) \\
							10進数を4ビットの2進数にコード化するような論理回路をエンコーダ(ENCODER)、
							4ビットの2進数コードをもとの10進数に戻すような回路をデコーダ(DECODER)といい、それぞれ基本論理素子を用いて構成することが出来る。
							\begin{itemize}
								\item 実習10 \\
									エンコーダ、デコーダをそれぞれ作成しなさい(10進"0"-"9")。また、作成した回路の動作確認を行い、真理地表を作成しなさい。
							\end{itemize}
					\end{enumerate}
\end{document}
