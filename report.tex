\documentclass[dvipdfmx]{jsarticle}
\begin{document}
	\title{組み合わせ回路と順序回路}
	\author{3EC 中野 将生}
	\maketitle
	\section{目的}
		本実験では論理回路実習装置(IFF-02)を用い、順序回路と組み合わせ回路の実習を行う。
		第一周では、基本的な論理演算解論についての実習を行い、論理素子の動作の確認を行うと共に、
		その応用として加算回路の実習を行い、組み合わせ回路についての理解を深めることを目的とする。
		\section{実習方法}
			\subsection{操作方法}
				\subsubsection{操作方法}
					\begin{itemize}
						\item 電源部(POWER) \\
							本装置の電源供給に関する操作箇所
						\item パルス発生器(PULSE GENERATOR)
							\begin{itemize}
								\item リセットパルス(RESET PULSE) \\
									プッシュスイッチを押している時間だけのパルス幅で、$A$端子からは正の、$\bar{A}$端子からは負の波形が出力される
								\item 単発クロックパルス(MANUAL CLOCK PULSE) \\
									プッシュスイッチを押すことにより、パルス幅約 $0.3 \mu sec$のパルスが、$A$端子からは正の、$\bar{A}$端子からは負の波形が出力される。
								\item クロックパルス (CLOCK PULSE)
									1 kHz端子からは1 kHzの、10 kHz端子からは10 kHzの方形波が出力される。
							\end{itemize}
						\item GND
							測定用アース端子。出力信号をオシロスコープ等で測定するときに、測定用アース端子を接続する。
						\item 表示器
							7セグメントタイプのLED表示器
							入力レベルが「1」のときは"1"、「0」のときは"0"と表示される。
					\end{itemize}
				\subsubsection{回路素子及び各回路素子の入・出力端子}
					\begin{itemize}
						\item 論理積(AND)
						\item 論理和(OR)
						\item 否定(NOT)
						\item 論理積の否定(NAND)
						\item 論理和の否定(NOR)
						\item エンコーダ(ENCODER) \\
						5入力OR1個、4入力OR4個、2入力OR1個から構成されている10進→2進の変換回路。
						\item デコーダ(DECODER) \\
						4入力AND10個で構成されている2進→10進の変換回路。
						\item 全加算回路 \\
						Exclusive ORを使用した全加算(FULL ADDER)回路。
						半加算回路またはExclusive ORとしても使用できる。
						\item R-Sフリップフロップ(R-S FLIP-FLOP)
						\item J-Kフリップフロップ(R-S FLIP-FLOP)
						\item シフトレジスタ(SHIFT REGISTER)
					\end{itemize}
\end{document}
